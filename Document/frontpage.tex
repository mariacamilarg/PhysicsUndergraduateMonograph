%Front page

\pagenumbering{roman}
%\pagenumbering{arabic}

\setcounter{page}{1}

\newpage

\thispagestyle{empty}
\begin{center}
  \vspace*{1cm}
 % {\Huge \bf Influence of galaxy rotation and outflows in the Lyman Alpha spectral line}
  {\LARGE \bf INFLUENCE OF GALAXY ROTATION AND OUTFLOWS IN THE LYMAN ALPHA SPECTRAL LINE}

  \vspace*{2cm}
  {\LARGE\bf Maria Camila \\Remolina Guti\'errez}

  \vspace*{1cm}
  {\LARGE\bf Advisor: \\ Dr. Jaime E. Forero-Romero}

  \vfill
  {\Large A monograph presented for the degree of\\
         [1mm] Physicist}
  \vspace*{0.2cm}
  
  % Put your university logo here if you wish.
   \begin{center}
   \includegraphics[scale=0.11]{figures/uniandes.jpg}
   \end{center}
  \vspace*{-0.4cm}   
  {\large Departamento de F\'isica\\
		  [-3mm] Facultad de Ciencias\\
          [-3mm] Universidad de los Andes\\
          [-3mm] Bogot\'a, Colombia\\
          [1mm]  November, 2015}

\end{center}

%Dedication page
\newpage
\thispagestyle{empty}
\begin{center}
\vspace*{2cm}
\textit{\LARGE {Dedicado a}}\\ 
a mi pap\'a quien nunca dejar\'a de ser mi h\'eroe, \\ 
a mi mam\'a quien nunca dejar\'a de cuidarme, \\
a Pipe por crecer conmigo, \\
a Sofi por ser mi mejor amiga, \\ 
a Juanda por ser mi eterno confidente, \\ 
a Jaime por ser mi gu\'ia y amigo, \\
a Lucas por cambiarme la vida, \\
...\\
a la Universidad de los Andes y la Beca Quiero Estudiar\\por cumplir mis sue\~nos. \\

\end{center}

%Abstract 
\newpage
\thispagestyle{empty}
\addcontentsline{toc}{chapter}{\numberline{}Abstract}
\begin{center}
  \textbf{\Large Influence of galaxy rotation and outflows in the Lyman Alpha spectral line}

  \vspace*{1cm}
  \textbf{\large Maria Camila Remolina Guti\'errez}

  \vspace*{0.5cm}
  {\large Submitted for the degree of Physicist\\ November, 2015}

  \vspace*{1cm}
  \textbf{\large Abstract}
\end{center}
Young galaxies in the Universe have a strong \lya emission caused by the ionized Hydrogen atoms in their interstellar medium. When the spectrum of a galaxy has an intense peak around the \lya natural frequency ($2.46\times 10^{15}$ Hz) it is called a Lyman Alpha Emitter (LAE). Typical LAEs are very distant ($z \gtrsim 2$). This makes that all the data astronomers can obtain from them is their spectra, and from there all the physical information of the galaxy must be derived. Trying to solve this task requires the creation of a simplified and solid model. In this monograph I propose to consider LAEs as a spherical distribution of Hydrogen atoms that undergoes a solid body rotation and a radial expansion. I simulate the effect of rotational velocity, outflow velocity and optical depth of the LAE in the outgoing spectra. The main conclusion is that this new model reproduces LAEs observed features in a clear way and with consistent physical parameters. However, proper observational fits are left for future work. This monograph accomplishes the objective of extracting as much information as possible from a LAE's \lya line.\\ 

%Abstract Español
\newpage
\thispagestyle{empty}
\addcontentsline{toc}{chapter}{\numberline{}Resumen}
\begin{center}
  \textbf{\Large Influencia de rotaci\'on y outflows en la l\'inea espectral Lyman Alpha}

  \vspace*{1cm}
  \textbf{\large Maria Camila Remolina Guti\'errez}

  \vspace*{0.5cm}
  {\large Presentado para el t\'itulo de Física\\ Noviembre, 2015}

  \vspace*{1cm}
  \textbf{\large Abstract}
\end{center}
Las galaxias j\'ovenes en el Universo tienen una fuerte l\'inea de emisi\'on \lya causada por los \'atomos de Hidr\'ogeno ionizados en su medio interestelar. Cuando el espectro de una galaxia tiene un pico intenso alrededor de la frecuencia natural \lya ($2.46\times 10^{15}$ Hz) se le llama una Lyman Alpha Emitter (LAE). Las LAEs t\'ipicas está\'an muy distantes ($z \gtrsim 2$). Esto hace que los datos que los astr\'onomos puedan obtener de ellas sean solo sus espectros, y es de estos que toda la información f\'isica de la galaxia debe ser derivada. Tratar de resolver esta tarea requiere crear un modelo s\'olido y simplificado. En esta monograf\'ia, yo propongo considerar las LAEs como una distribuci\'on esf\'erica de \'atomos de Hidr\'ogeno, que rota como un cuerpo r\'igido y se expande radialmente. Yo simulo los efectos de velocidad rotacional, velocidad de outflows y profundidad \'optica de la LAE en el espectro de salida. La conclusi\'on principal es que este nuevo modelo reproduce las caracter\'isticas principales de LAEs observadas. Si embargo, fits observacionales se dejan para trabajo futuro. Esta monograf\'ia logra el objetivo de extraer la mayor informaci\'on posible de la l\'inea \lya de una LAE.\\ 
 

%Declaration page
\chapter*{Declaration}
\addcontentsline{toc}{chapter}{\numberline{}Declaration}
The work in this monograph is based on research carried out at the Physics Department of Universidad de los Andes, Colombia. No part of this work has been submitted elsewhere for any other degree and it is all my own work unless referenced to the contrary in the text. 

\vspace{4in}
\noindent \textbf{Copyright \copyright\; 2015 by MARIA CAMILA REMOLINA GUTI\'ERREZ}.\\
``The copyright of this monograph rests with the author.  No quotations
from it should be published without the author's prior written consent
and information derived from it should be acknowledged''.

%Acknowledgements
\chapter*{Acknowledgements}
\addcontentsline{toc}{chapter}{\numberline{}Acknowledgements}
I acknowledge my advisor Jaime Forero-Romero for all his guidance and teachings along my time at the university. Also to Juan Nicolas Garavito-Camargo for becoming my academic older brother. Finally, to AstroAndes for all the wonderful group meetings. \\

I acknowledge the Scientific Computational Cluster of Universidad de los Andes because all the simulations were run in its processors. Also, most of my code benefits from the work of the IPython and matplotlib communities \cite{IPython,matplotlib}.\\

%tableofcontents
\tableofcontents
\clearpage