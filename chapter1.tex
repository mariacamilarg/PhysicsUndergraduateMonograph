\setcounter{equation}{0}
\chapter{Introduction}

\section{The Lyman Alpha emission line}

The Lyman Alpha ( \lya) emission line is the spectral line produced by an excited hydrogen atom when its electron jumps from the second energy level to the first one. This fall causes an emission of a photon with a corresponding wavelenght of $1215.67$ $\AA$. The whole Lyman series was discovered in 1906 by the American physicist Theodore Lyman. \\

The derivation from this series comes from Niels Bohr's model in which the energy of each level is given by Eq. \ref{eq:energy}; where $m$ and $e$ are the mass and charge of the electron, respectively, $\epsilon_0$ is the vacuum permittivity, $\hbar$ is the reduced Planck constant and $n$ is the level. 

\begin{equation}
\label{eq:energy}
E_n = -\frac{me^4}{2(4\pi\epsilon_0\hbar)^2}\frac{1}{n^2}
\end{equation}

When the electron falls from one energy level to the other, it emits a photon with a certain wavelength, determined by Eq. \ref{eq:wavelength}; where $h$ is the Planck constant, $c$ is the speed of light, $E_i$ is the energy at the initial level $i$ and $E_f$ is the energy at the final level $f$.

\begin{equation}
\label{eq:wavelength}
\lambda = \frac{hc}{E_i-E_f}
\end{equation}

For the Lyman series the final level is always $n=1$, and for the \lya line, the initial level is $n=2$. After plugging this numbers, a wavelength of $1215.67\AA$ is obtained. This will be the transition we're interested in.\\

In 1967 Partridge et al. \cite{PartridgePeebles} predicted that young galaxies would have a strong \lya  emission. Nowadays galaxies selected using the \lya line are known as Lyman Alpha Emitters (LAEs). Since the first observed LAE by Djorgovski et al. \cite{DjorgovskiThompson} different teams have observed several LAEs (\cite{Rhoads00}, \cite{Gawiser2007}, \cite{Koehler2007}, \cite{Ouchi08}, \cite{Yamada2012}, \cite{Schenker2012}, \cite{Kulas12}, \cite{Yamada2012}, \cite{Chonis2013}, \cite{Finkelstein2013}, \cite{Ostlin14}, \cite{Hayes2014}, \cite{Faisst2014}, \cite{Fumagalli2015}) specially at $z\geq2$ since it is when the  line is redshifted into the optical regime. LAEs became then a tool to explore the extragalactic Universe and with upcoming telescopes such as the James Webb Space Telescope, new ones are going to be discovered with a better resolution and at higher redshifts. This creates a clear motivation to model them and observe them. \\

As for motivation, these observations have a direct impact in studying the reionization epoch (\cite{review}), properties of the interstellar medium (ISM) and the intergalactic medium (IGM) (\cite{Behrens13}, \cite{DijkstraKramer}), constraining star formation rates of high redshift galaxies, understanding galaxy luminosity functions (\cite{Max}) and studying the large scale structure of the Universe. In all of these studies, an understanding of the processes that model the morphology and radiate transfer process behind the \lya line, is required. To fully understand the observed spectra of the LAEs, these galaxies must be modeled. \\

\section{Existing models of Lyman Alpha Emitters}

The resonant nature of the \lya line makes modelling it a challenging task, but constraining the nature of LAEs using the line is a common and strong objective. Analytical solutions for the outcoming spectra in simple ISM static geometries have been derived (\cite{Adams72}, \cite{Harrington73}, \cite{Neufeld90}, \cite{Dijkstra06}). Radiative transfer codes (\cite{DijkstraKramer}, \cite{Laursen09}, \cite{Verhamme06}, \cite{CLARA}) have been developed in order to understand the effect of the gas kinematics in the \lya line. Special attention have been devoted to  the effects of clumpy media (\cite{Hansen06}) and expanding/contracting shell/spherical geometries started to be studied (\cite{Ahn03}, \cite{Verhamme06}, \cite{Dijkstra06}). Hydrodynamic simulations have studied the outcomming spectra of LAEs in large scale simulations \cite{Forero12}. Escape of \lya photons at the line center is also a proposed model that fits the observations in a more accurate way (\cite{Martin2015}, \cite{Garavito14}, \cite{Neufeld91}). And Monte Carlo codes have been used in hydrodynamic simulations to study in detail individual galaxies (\cite{Laursen09}, \cite{Barnes11}, \cite{Verhamme12}, \cite{Yajima12}).\\

Special attention have been devoted to model the presence of outflows in these galaxies, motivated by previous observational studies. Outflows are a consequence of the interstellar medium (ISM) being ejected from the galaxy due to supernova explosions. Here different models have attempt to simulate more realistic situations involving shell models and cavities (\cite{Behrens2014}). Blue wings and bumps have been modeled when the outflows regulating the escape of \lya photons are still engulfed within a static interstellar medium (\cite{Chung2015}). Verhamme et al. \cite{Verhamme06} created an expanding shell model that despite its geometric simplicity, has been able to fit several \lya profiles including: observational, as the ones studied by Hashimoto et al. \cite{Hashimoto2015} who reproduced the sources \lya lines and calculated their outflow velocities, and simulated, as the ones created by Gronke et al. \cite{Gronke2015} who determined if degeneracies exist between the different shell model parameters. Also, Orsi et al. \cite{Orsi12} creates a wind shell model that could be interpreted as an expanding sphere. Rivera-Thorsen et al. \cite{Rivera-Thorsen2015} found that no one single effect dominates in governing \lya radiative transfer and escape, and that \lya peak velocities are consistent with a simple model of an intrinsic emission line overlaid by a blueshifted absorption profile from the outflowing wind.\\

Despite the fact that outflows have been broadly studied, rotation should also be present in these galaxies. Recently, a rotation model, created by Garavito et al. \cite{Garavito14} models a rotating spherical galaxy with homogeneous gas mixture and analyzes that impact in the \lya line. However not a lot of attention has been payed to this effect in LAEs.\\

The joint effect of the above properties should have a direct impact on the morphology of the \lya line. This is the subject of this work. \\

\section{Radiative Transfer Code}

A radiative transfer process is a ``physical phenomenon of energy transfer in the form of electromagnetic radiation. The propagation of radiation through a medium is affected by absorption, emission and scattering processes.'' (\cite{Ryden2010}). This phenomenon is described by the equation of radiative transfer, Eq. \ref{eq:radiative}; where $I_{\nu}$ is the spectral radiance, $\Omega$ is the solid angle, $k_{\nu,s}$ is the scattering cross section, $k_{\nu,a}$ is the absorption cross section, and $j_{\nu}$ is the emission coefficient. (\cite{Chandrasekhar1950})

\begin{equation}
\label{eq:radiative}
\frac{1}{c}\frac{\partial}{\partial t}I_{\nu} + \hat{\Omega}\cdot\nabla I_{\nu} + (k_{\nu,s} + k_{\nu,a})I_{\nu} = j_{\nu} + \frac{1}{4\pi c}k_{\nu,s}\int_{\Omega}I_{\nu}d\Omega
\end{equation}

The radiative transfer phenomenon describes, among several physical processes, the scattering of a photon in a LAE that has to pass through the galaxy while being absorbed and re-emitted until it escapes. This effect is the one we are interested in. However, analytic solutions for the previous equation have not been derived for non-simple cases, and for this particular situation numerical methods, achieved through computation, are required. \\

This problem creates a motivation for radiative transfer codes to be implemented. In this monograph a modified version of the code CLARA (Code for Lyman Alpha Radiation Analysis) made by \cite{CLARA} will be used. The code works as follows.\\

We propose a simplified model in which the galaxy is modeled as an sphere, a homogeneous mixture of dust and hydrogen, undergoing solid-body rotation and that is expanding in the radial direction due to outflows. There is a central emission in the galaxy of a large number ($\sim 100000$) of photons. In CLARA, one by one, each photon starts in a random direction in the $xyz$ plane with the natural \lya frequency. It continues that path until it encounters a hydrogen atom that absorbs it. It is then re-emitted in another direction and with a different frequency that depends of the velocity, both radial and tangential, of the atom. As there is no dust present in the simulation, the photon is never absorbed by it and continues in this random walk until it escapes from the galaxy. At each point of the walk, CLARA tracks the photon's position, direction and frequency. The final output is a table of these final values for each photon. \\

%¿PREGUNTA?: la tasa de escape (100%) es eso porque el no hay polvo? yo puse it can be re-emitted porque no estaba segura si asegurar que como no hay polvo siempre será re emitido. 

%¿PREGUNTA?: CLARA mas a fondo?

\section{Monograph Overview}
This paper is structured as follows. In Chapter 2 we explain in detail the model of rotation and outflow that we use. In Chapter 3 we present the results of our model, specially how the morphology of the line changes with the free parameters. In Chapter 4 we compare our results with a recent observation of a LAE. In the latest Chapter, 5 we present our conclusions and possible future work. Finally, in the Appendix, we present the results of a \lya modeled only with the rotation analytical solution by Garavito et al. \cite{Garavito14} that then is filtered by a Verhamme et al. \cite{Verhamme06} expanding shell.\\